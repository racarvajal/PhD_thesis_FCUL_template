\pagestyle{MyThesisStyle}
\chapter[\MakeUppercase{Introduction}]{Introduction}\label{chap:chapter_01}

\section[\MakeUppercase{First section of Introduction}]{First section of Introduction}\label{sec:first_section}
\sectionmark{\MakeUppercase{First example section}}

The text of this chapter is shown as written. Additionally, it is possible to include glossary entries, such as \gls{nasa}. The definitions for such entries must be included in the file \texttt{glossaries.tex} following the instructions from the packages \href{https://www.ctan.org/pkg/glossaries}{\texttt{glossaries}} and \href{https://ctan.org/pkg/glossaries-extra}{\texttt{glossaries-extra}}. The first time they are used, the full description is included. For additional uses, only the acronym is shown, as in \gls{nasa}.

It is also possible to use symbols in the same way. For instance, to include something as \gls{hubble_constant}. Then, these symbols and acronyms will be shown in the preambles.

References can also be used in the document. In-text names can be added, such as \textcite{MartinThrones}. It is also possible to show them in parentheses \parencite[such as:][]{whitman1855leaves}. Here, the commands \texttt{parencite} and \texttt{textcite} are preferred since they are more powerful than the older \texttt{citet} and \texttt{citep}.

An example figure is shown in Fig.~\ref{fig:example_figure_01}. The size it will have when displayed can be selected.

\begin{figure}[htbp]
    \centering
    \includegraphics[width=0.65\columnwidth]{example-image}
    \caption[First example image]{First example image in this document. There are several ways in which a figure can be added.}
    \label{fig:example_figure_01}
\end{figure}

Including more than one image per figure is also possible thanks to the package \href{https://ctan.org/pkg/subcaption}{\texttt{subcaption}}. Figure~\ref{fig:example_subfigure} shows and example of its implementation, where Figs.~\ref{fig:first_subfigure} and \ref{fig:second_subfigure} can be referenced individually.

\begin{figure}[tbhp]
    \centering
    \begin{subfigure}[t]{0.40\textwidth}
        \centering
        \caption{First subfigure}
        \includegraphics[width=0.99\linewidth]{example-image}
        \label{fig:first_subfigure}
    \end{subfigure}
    \begin{subfigure}[t]{0.40\textwidth}
        \centering
        \caption{Second subfigure}
        \includegraphics[width=0.99\linewidth]{example-image}
        \label{fig:second_subfigure}
    \end{subfigure}
    \caption[Example of figure with two images]{Example of figure with two images. They can be labelled and called individually.}
    \label{fig:example_subfigure}
\end{figure}

\section[\MakeUppercase{Second section of Introduction}]{Second section of Introduction}\label{sec:second_section}
\sectionmark{\MakeUppercase{Second example section}}

As expected more sections can be added in one chapter. In it, it is possible to include, for instance, a table. The preferred way to do it is through the packages \href{https://ctan.org/pkg/threeparttable}{\texttt{threeparttable}} and \href{https://ctan.org/pkg/threeparttablex}{\texttt{threeparttablex}}. They allow, for instance, including footnotes directly below the table. Also, the package \href{https://ctan.org/pkg/booktabs}{\texttt{booktabs}} creates the commands \texttt{toprule}, \texttt{midrule}, \texttt{bottomrule}, and \texttt{cmidrule}. Additionally, the use of the package \href{https://ctan.org/pkg/siunitx}{\texttt{siunitx}} makes easier to align quantities in different columns. An example is shown in Table~\ref{table:example_table_a}.

\begin{table}[tbhp]
    \setlength{\tabcolsep}{4.5pt}
    \centering % used for centering table
    \caption[First example table]{First example table. It highlights the use of the packages \texttt{sinuitx}, \texttt{threeparttable}, and \texttt{booktabs}.}             % title of Table
    \label{table:example_table_a}      % is used to refer this table in the text
    \begin{threeparttable}
    \begin{tabular}{c S[table-format=2.2+-1.2] S[table-format=2.2+-1.2] S[table-format=2.2+-1.2] S[table-format=2.2+-1.2]} % centered columns (5 columns)
    \toprule
    \multicolumn{5}{c}{First part} \\
    Name\tnote{a}   & {Column 1}    & {Column 2}     & {Column 3}    & {Column 4} \\
                    & {(unit 1)}    & {(unit 2)}     & {(unit 3)}    & {(unit 4)} \\
    \midrule
    Row A           & 34.10(2.1)    & 78.78(2.2:2.1) & 33.28(3.4)    & 30.51(2.3) \\
    Row B           & 51.80(1.3)    & 49.71(3.1:1.5) &  8.77(1.2)    &  7.12(1.1) \\
    Row C           & 65.21(4.0)    &  1.30(0.2:0.1) & 67.68(3.3)    &  3.43(0.9) \\
    Row D\tnote{b}  &  3.14(0.1)    & 75.76(1.1:1.9) & 77.34(5.1)    & 22.62(2.3) \\[0.5em]
    \toprule
    \multicolumn{5}{c}{Second part} \\
    Name\tnote{a}   & {Column 1}    & {Column 2}     & {Column 3}    & {Column 4} \\
                    & {(unit 1)}    & {(unit 2)}     & {(unit 3)}    & {(unit 4)} \\
    \midrule
    Row A           & 63.59(4.3)    & 25.47(5.5:2.3) & 33.00(3.3)    & 16.52(1.2) \\
    Row B           & 46.80(2.3)    & 28.22(1.1:1.1) &  9.59(1.2)    & 33.44(4.2) \\
    Row C           & 24.63(1.1)    &  7.76(0.3:0.2) & 28.80(2.6)    &  0.32(0.1) \\
    Row D\tnote{b}  &  9.73(0.9)    & 51.02(2.3:2.1) & 38.72(7.4)    & 38.37(0.5) \\
    \bottomrule
    \end{tabular}
    \begin{tablenotes}
        \item [a] Footnote for the first column of the table.
        \item [b] Footnote for a particular row.
    \end{tablenotes}
    \end{threeparttable}
\end{table}

As expected, mathematical expressions can also be included in this document.

\begin{equation}\label{eq:example_equation}
    \sin^2(a)+\cos^2(a) = 1\,,
\end{equation}

\noindent where each element of Eq.\ref{eq:example_equation} can be defined. Also, matrices can be included, such as the following.

\begin{equation}\label{eq:example_matrix}
    \begin{pmatrix}
      (1 - p)^{2} & p (1 - p)\\
      (1 - p) p   & p^{2}
    \end{pmatrix}\,.
\end{equation}

Another relevant feature is the use, as mentioned earlier, of the package \href{https://ctan.org/pkg/siunitx}{\texttt{siunitx}}. Adding the needed parameters in the preamble, it is possible to create new units and quantities that can be used. For instance, masses can be expressed as $\mathrm{M}_{\sun} \sim \qty{2e30}{\kg}$ and then use new units as $M_{a} = \qty{1e6}{\msun}$. Checking the file \texttt{mythesis.cls} additional units (and their definitions) can be found (e.g. $\unit{\milli\jansky}$, $\unit{\deg^{2}}$, $\unit{\angstrom}$). In the same way, it is also possible to include angular distances, such as $\ang[angle-symbol-over-decimal]{2;23;15.1}$, and celestial coordinates: right ascension $\ang[angle-symbol-degree=h, angle-symbol-minute=m, angle-symbol-second=s]{12;12;00}$ and declination $\ang{-20;00;00}$.